%!TEX root = ./seminar.tex
\section{Smartphones as Digital Health Inhibitors}
\label{sec:smartphoneIssues}
\subsection{Declining Attention Span}
The mantra of screens making us dumb has been around for a while now. Ever since television became available in basically every household, the fight for our attention is ever present. With smartphones as always-on and always-something-new-to-check devices, the call for more moderation and mindful consumption of the provided entertainment is important. Still, the mere presence of smartphones uses up our cognitive resources, even in individuals that can maintain sustained attention and resist the temptation to check their smartphone \cite{ward2017brain}. The underlying problem is rooted in our biology in form of overloaded dopamine receptors and further inhibited by our very pro-social society. The former makes us more prone to seeking instant gratification and in a world where a laugh or piece of new information is only a scroll away, resisting this temptation is not easy. This leads to our brain being flooded with the neurotransmitter dopamine. The brain tries to protect itself from this overload by reducing the sensitivity of the responsible receptors, thereby increasing the threshold for further stimuli. At the same time, we still craves the good feeling induced by dopamine. The result is a constant need for further small stimuli, so we scroll on \cite{nieoullon2002dopamine,dopamineRole}. The later is a byproduct of our ability to communicate very effectively with one another. Evolutionary, we tend to want to know what other people saw and experienced. This sharing of information, e.g. where the next berry bush is or where the lion patrols, was very effective for our collective survival \cite{sapiens}. But our brains do not evolve as fast as our society does, and therefore this interest in gossip, stories and other pieces of information takes a huge toll on maintaining a healthy relationship with technology. The largest and still mostly unknown impact of this behavior is found in still developing brains \cite{crone2018media}.
\subsection{Social Media Impact on Young Adults}
Our brains need a lot more time to mature then we think. The prefrontal cortex, responsible for well evaluated decisions is only fully developed by age 25 \cite{prefrontalCortexDev}. Therefore, adolescents are highly sensitive to acceptance and rejection through social media. This may make them specifically reactive to emotion-arousing media The so-called Generation of Digital Natives accesses media-related content for roughly 6-9 hours a day (United States) \cite{crone2018media}. As the content is readily available at the swipe of a finger, many young people struggle with embracing delayed gratification in fear of missing out \cite{oberst2017negative}. At the same time the question must be ask, whether emotional arousal may influence the use of health trackers in a negative way. With all the statistics of ones own body available, anxieties can emerge and toss a heavy toll on young peoples self esteem. As we know, this can have life long effects on an individuals well-being \cite{mykletun2006mortality}.
\subsection{Conclusion of Mobile Health(care)}
Concluding from Section~\ref{sec:smartphoneChances} and Section~\ref{sec:smartphoneIssues}, smartphones can be attributed to have game-changing influence in tackling emerging issues in digital and mobile healthcare. While they surely have their downsides on mental health when used incorrectly, they offer a wide range of possibilities for health improvements in all demographics and health conditions. With investment booming \cite{safavi2019top} and research accumulating \cite{firth2016ecological}, public health will hopefully see further improvements that make the healthcare environment more efficient in the near future.
