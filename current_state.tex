%!TEX root = ./seminar.tex
\section{The Current State of Digital Health(care)}
\subsection{Costs}
Looking at the current state of healthcare in general around the world, we see a trend of increasing costs year by year. Estimations consider a global anual cost of \$8.7 trillion by 2020 in contrast to \$7 trillion in 2015. This corresponds to a steady 10\% of the GDP worldwide on average \cite{EIU2016}. Exspecially in North America and Europe this is due to increasing rates of multiple chronic conditions and age related diseases \cite{sambamoorthi2015multiple}. Another major contributor to rising healthcare cost are "preventable adverse drug events associated with inpatient injectable medications" \cite{lahue2012national}; With an estimated 7 million patients suffering the consequences in the United States they contribute to 7000 deaths and almost \$21 billion in direct medical costs across all care settings anually \cite{prevMedErrors}.
\subsection{Efficiency}
About 30\% of global data volume is generated by the healthcare industry \cite{gopal2019digital}, but healthcare data is among the slowest to be digitalized \cite{industryDigitalization}. This leaves us with large quantites of data in unstructured and analog formats. The process to digitalize them is prone to errors as well as time intense. Time, our healthcare systems do not have, as more and more patients with chronic diseases require said time \cite{ostbye2005there}. This results in less time healthcare professionals can spend with their patients \cite{fuchtbauer2013emergency} and therefore a decline in patient satisfactory \cite{gross1998patient}. With more frequent interactions due to chronic disease this inhibits a good relationship between doctors and their patients.
\subsection{Risks}
Healthcare enviroments should not allow even the smallest margin of error when it comes to the health outcome of patients. Still, illegible handwriting is far to common among phycicians such that about a sixth of doctor's notes are unclear to others \cite{rodriguez2002illegible}. With the already seen costs of medication errors this is a huge risk factor for the healthcare enviroment.

