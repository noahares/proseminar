%!TEX root = ./seminar.tex
\section{Introduction}
Healthcare environments are slow to adapt digitization \cite{industryDigitalization} and therefore struggle with an onslaught of cost \cite{EIU2016}, efficiency \cite{gopal2019digital} and risk related issues \cite{rodriguez2002illegible}. Possible drivers for a much needed paradigm shift are smartphones in combination with wearable sensors that are currently available to the public in an ever increasing variety. New technologies allow for better monitoring of health markers and provide advice to the user. With this, big tech players like Apple and Google entered the healthcare market and provide health services based on massive amounts of user data \cite{appleHealth}. Smart AI-based algorithms process this data and thereby improve health outcomes \cite{villar2015improving}. AI is also emerging in the public health sector in documentation and image procession \cite{kiKroenung}. The problem with this collection of user and patient data lies in security. Here, guidelines and laws have to be passed that allow for a secure and efficient usage of said data. A very likely scenario for the future of digital healthcare bases value on health outcomes and not just the provided service, tailoring care to individual needs and making people "more healthy" in the long term \cite{putera2017redefining}. This includes education of medical professionals \cite{adams2010nutrition} as well as the public about preventive measurements for better health with evidence based recommendations, collected with the help of AI in clinical trials and studies \cite{rehme2014identifying}.

