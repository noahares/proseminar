%!TEX root = ./seminar.tex
\section{Conclusion}%
\label{sec:conclusion}

The current state of healthcare environments around the world in terms of digitization is far from ideal. This makes them very costly, inefficient and even ineffective. Chronic diseases are a real time and money threat to these systems. Smartphones, at the moment, are a very promising technology to alleviate individuals from regular routine visits at their doctors office. They can be used to communicate with worn sensors to analyze a multitude of health metrics. Recent wearables even allow things like continuous monitoring of blood glucose levels right at the patients wrist and without the need to draw blood. Another very wide application of smartphones in combination with wearables is health tracking in overall healthy people and athletes, providing them with real time information and advice that can be reacted to accordingly. Of course, smartphones in general also have their downsides. They pose a long lasting threat to the mental health of young people through social media and potential competitiveness in health markers. Therefore they must be used responsibly to ensure positive health outcomes from their use. In a more collective setting it is not clear, whether public healthcare providers can make to necessary improvements in digitization in the near future, but big players like Apple and Google entered the healthcare market recently. Their approach takes the gathering of big amounts of users health data as a foundation to provide individual healthcare services to their users. Their underlying algorithms are often AI-based. AI is also emerging in the practices of public healthcare providers, but mostly to a lesser extend depending on the country. Its widest application is found in documentation assistance and image processing in low-risk cases. Smart implants are another emerging technology that can treat and monitor chronic conditions like diabetes in a very patient friendly way without painful injections of insulin. A very promising approach to future healthcare is value-based healthcare. The underlying philosophy values patients overall health including mental and social health. This also means embedding more preventive measurements to lower peoples risks of developing incommutable diseases like Alzheimer's disease, diabetes and cancer. One of these measurements is healthcare through education. This means providing the public with clear recommendations regarding nutrition and exercise as well as laying out evidence-based information to different lifestyles, so individuals can educate themselves about a healthy way of living. Additionally, medical professionals also need to be educated in medical school about preventive measurements like nutrition that even allow for a very cost efficient treatment in cases where medication is not a long term solution that people can live happily with. To introduce such treatments clinical trials with AI-based evaluation can be used to ensure the best possible completeness of information.
