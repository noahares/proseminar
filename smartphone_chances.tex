%!TEX root = ./seminar.tex
\section{Smartphones as Digital Healthcare Devices}
\subsection{Trackers}
Smartphones are more and more incorperated in our day to day activities. From waking us up in the morning to social media being the last thing we close before we do the same to our eyes. Having these portable and computionally powerful devices everywhere we go allows monitoring of various events throughout the day. In the last couple of years fitness tracker are on the rise to replace a traditional watch on peoples wrists. The standard functionality of even the cheapest models include a heartrate monitor, step counter and of course a clock. This seamingly simple collection of data turned out to be of immense value in a case study from 2016 where a 42-year-old man was presented to the emergency department with newly diagnosed atrial fibrillation and otherwise normal physical examination. By looking at the synchronized data on his smartphone provided by his wrist-worn activity tracker, the medical staff was able to identify the onset of the arrhythmia as within the previous 3 hours, permitting eloctrocardioversion which is only applicable if the onset was less than 48 hours ago. The single correction of cardiovascular activity left the patient with with a normal sinus rythm at his follow-up appointment \cite{rudner2016interrogation}. \\.
\subsection{Imidiate Feedback}
\subsection{Integrating Further Medical Devices}

