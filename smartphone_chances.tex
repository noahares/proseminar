%!TEX root = ./seminar.tex
\section{Smartphones as Digital Healthcare Devices}
\subsection{Trackers}
Smartphones are more and more incorperated in our day to day activities. From waking us up in the morning to social media being the last thing we close before we do the same to our eyes. Having these portable and computionally powerful devices everywhere we go allows monitoring of various events throughout the day. In the last couple of years fitness tracker are on the rise to replace a traditional watch on peoples wrists. The standard functionality of even the cheapest models include a heartrate monitor, step counter and of course a clock. This seamingly simple collection of data turned out to be of immense value in a case study from 2016 where a 42-year-old man was presented to the emergency department with newly diagnosed atrial fibrillation and otherwise normal physical examination. By looking at the synchronized data on his smartphone provided by his wrist-worn activity tracker, the medical staff was able to identify the onset of the arrhythmia as within the previous 3 hours, permitting eloctrocardioversion which is only applicable if the onset was less than 48 hours ago. The single correction of cardiovascular activity left the patient with with a normal sinus rythm at his follow-up appointment \cite{rudner2016interrogation}. \\
Another variety of trackers that is now just emerging includes monitoring of blood glucose levels. This allows diabetics to evaluate their blood glucose with the swipe of a finger and without the need to draw blood at all \cite{glucoseTracker}. We will talk about this technology again in a later section.
\subsection{Immediate Feedback}
With the ability to monitor ones own health markers, as well as activities and react to them immediately is a huge and time saving advantage over traditional healthcare environments where you either need to visit a doctor or have a large amount of equipment like blood pressure or blood glucose measurement kits to get information about your current state of well being. Especially with the rise of chronic conditions and their quickly accumulating costs for patients and the healthcare system we saw earlier, this allows doctors to spend more time treating with more urgent matters and patients to enjoy life more with the time they save. Talking about enjoyment, it has been reported that people actively use the information provided by their health trackers to improve their current state of well being. Whether this includes getting into motion by trying to complete a daily step goal \cite{rasche2015activity} or calming themselves by observing an elevated heart rate induced by stress \cite{mayya2015continuous}.
\subsection{Integrating Further Medical Devices}

