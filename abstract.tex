%!TEX root = ./seminar.tex
\begin{abstract}
    Healthcare environments around the world seem to have missed out on adapting to digitization. With inefficient and error-prone practices in documentation and treatment, it has become very costly to provide healthcare. As human lifespan further increases and chronic diseases become more and more common, some bold changes have to be made to ensure well being under these conditions - or prevent them in the first place. This work will first explore the current state of healthcare in general. Then smartphones, wearable sensors and other mobile solutions for individual health monitoring and advisory will be evaluated and their potential health hazards pointed out. This includes technology that has established itself to the masses and just emerging solutions that are still in development. Further, the concerns of secure management of patient data will be dealt with. It shall be elaborated, what has to be done to lay a solid foundation of security and accessibility of patient data to benefit from the application of Artificial Intelligence (AI) in healthcare. Afterwards, patient-centered healthcare is meet again with an outlook at individually tailored care and the use of minimally intrusive smart implants for monitoring and treatment of conditions like diabetes. Finally, value-based healthcare will be touched, elaborating on the underlying philosophy and how it can improve health outcomes. Another important aspect of value-based healthcare is healthcare improvement through education. The argument here shall be, that while people have more access to (health) information than ever, healthcare providers need to establish clear recommendations and lay out evidence based material regarding safety of various health-affecting lifestyles.
\end{abstract}
